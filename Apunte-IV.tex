\documentclass[a4paper, 10pt]{report}
\usepackage{common}
\usetikzlibrary{petri,positioning}

% Cuerpo
\begin{document}

% Carátula
\title{
    IV\\
    Programación Paralela\\
    \large Clase 10: Introducción a la Programación Paralela
}
\author{Ramiro Martínez D'Elía}
\date{2021}
\maketitle

% Índice
\tableofcontents

\chapter{Conceptos}

\section{Programa paralelo}

Tipo de programa, concurrente, escrito para resolver un problema en menos tiempo que su análogo secuencial. El objetivo principal es reducir el tiempo de ejecución, o resolver problemas más grandes o con mayor precisión en el mismo tiempo.

Un programa paralelo puede escribirse usando variables compartidas o pasaje de mensajes. La elección la dicta el tipo de arquitectura.

Para determinar si una solución paralela tiene mejores prestaciones que su análogo secuencial, existen nociones como: \textbf{\emph{speedup}}, \textbf{\emph{eficiencia}}, \textbf{\emph{escalabilidad}}, y \textbf{\emph{ley de Amndhal}}.

\section{Speedup}

La métrica de speedup (o eficiencia) \textbf{evalúa el tiempo total de ejecución} de un programa. El speedup está dado por la siguiente fórmula:

\begin{center}
$S = Speedup = T_1 / T_p$
\end{center}

Donde $T_1$ es el tiempo de una solución secuencial ejecutando en 1 CPU y $T_p$ es el tiempo de una solución paralela con $p$ CPUs. 

Del resultado obtenido, podemos concluir las siguientes afirmaciones:

\begin{itemize}
    \item Si $S = P \rightarrow $ speedup \textbf{\emph{lineal}} (o \textbf{\emph{perfecto}}).
    \item Si $S < P \rightarrow $ speedup \textbf{\emph{sublineal}}
    \item Si $S > P \rightarrow $ speedup \textbf{\emph{superlineal}}.
\end{itemize}

Por ejemplo, si el tiempo de una solución secuencial es de 600 segundos y el de una solución paralela es de 60 segundos utilizando 10 CPU.

\begin{center}
    $Speedup = T_1 / T_p = 600 / 60 = 10$

    Como $S = P \rightarrow$ speedup lineal
\end{center}

\section{Eficiencia}

Esta métrica permite conocer \textbf{qué tan bien aprovechará procesadores extras} un programa paralelo. La eficiencia, está dada por la siguiente fórmula:

\begin{center}
    $E = Eficiencia = Speedup / p$
\end{center}

Donde $p$ es la cantididad de CPUs. Del resultado obtenido, podemos concluir las siguientes afirmaciones:

\begin{itemize}
    \item Con speedup perfecto $\rightarrow E = 1$.
    \item Con speedup sublineal $\rightarrow E < 1$.
    \item Con speedup superlineal $\rightarrow E > 1$.
\end{itemize}

Por ejemplo, si el tiempo de una solución secuencial es de 600 segundos y el de una solución paralela es de 60 segundos utilizando 10 CPU.

\begin{center}
    $Speedup = T_1 / T_p = 600 / 60 = 10$
\end{center}
\begin{center}
    $Eficiencia = Speedup / P = 10 / 10 = 1$
\end{center}

Como el speedup obtenido fue lineal ($S=P$), entonces se cumplió que $E=1$.

\section{Escalabildiad}

Las métricas de \textbf{speedup} y \textbf{eficiencia}, son \textbf{relativas}. Ellas dependen del número de procesadores, el tamaño de los datos y el algoritmo utilizado. Por esto, de forma general, se dice que \textbf{un programa paralelo es escalable si; su eficiencia se mantienen constante para un rango amplio de CPU}.

Por ejemplo: una solución es paralelizada sobre $P$ procesadores de dos maneras. En la primera, el speedup está regido por la función $S = P-5$. Mientras que, en la otra $S = P/2$. ¿Qué solución se comportará más eficientemente al crecer $P$?

{\renewcommand{\arraystretch}{2}%\
\begin{center}
\begin{tabular}{p{2cm} p{4cm} p{4cm}}
\textbf{P} &  \textbf{S = P - 5} & \textbf{S = P / 2} \\
6 & $(6-5)/6=1,66$ & $(6/2)/6=0,5$ \\
\hline
10 & $(10-5)/10=0,5$ & $(10/2)/10=0,5$ \\
\hline
20 & $(20-5)/20=0,75$ & $(10/2)/20=0,5$ \\
\end{tabular}
\end{center}}

La solución regida por el speedup $S = P/2$ es la más escalable. Ya que, el valor de $E$ se mantiene constante al incrementar el número de procesadores.

\section{Ley de Amdhal}
   m            
Un programa típico consta de 3 (tres) etapas; entrada de datos, cómputo y salida de resultados. 

La ley de Amdhal postula que; para todo algoritmo \textbf{existe un speedup máximo alcanzable}, independiente del número de procesadores. Ese valor dependerá de la cantiadd de código paralelizable.

\begin{center}
    $Limite = 1 / 1 - Paralelizable$
\end{center}

Por ejemplo: suponga un programa secuencial donde las etapas (1) y (2) consumen cada una el 10\% del tiempo de ejecución y no pueden ser paralelizadas. La etapa (3) consume el 80\% restante. Si el programa tarda 100 unidades de tiempo ($ut$), ¿cuál es el limite de mejora alcanzable?

\begin{center}
    $Limite = 1 / 1 - 0,8 = 1 / 0,2 = 5$
\end{center}

El mejor speedup alcanzable es 5. Esto quiere decir que; el tiempo de ejecución, del algoritmo paralelizado, será como máximo hsta 5 veces más rápido ($100/5 = 20 ut$)

\section{Granularidad}

Cuando el número de procesadores crece, la carga de cómputo en cada unidad tiende a decrecer y las comunicaciones aumentan. Esta relación se conoce como \textbf{granularidad}.

La granularidad de una aplicación o máquina paralela está dada por la relación entre el cómputo promedio y la comunicación promedio que realice.

Si la granularidad del algoritmo difiere a la de la arquitectura, normalmente se tendrá pérdida de performance.

{\renewcommand{\arraystretch}{2}%\
\begin{center}
\begin{tabular}{p{3cm} p{6cm} p{6cm}}
& \textbf{Grano grueso} & \textbf{Grano fino} \\
\textbf{Arquitectura} & Pocos CPU, con mayor capacidad de cómputo. & Muchos CPU, con menor capacidad de cómputo. \\
\hline
\textbf{Aplicación} & Pocos procesos, con responsabilidades más amplias $\rightarrow$ menor comunicación. & Muchos procesos, con responsabilidades más chicas $\rightarrow$ mayor comunicación. \\
\end{tabular}
\end{center}}

\end{document}