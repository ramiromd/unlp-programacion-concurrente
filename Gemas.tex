\documentclass[a4paper, 10pt]{report}
\usepackage{common}

\begin{document}
    
% Carátula
\title{
    \faIcon[regular]{gem}\\
    Programación Concurrente\\
    \large Preguntas de final
}

\author{Ramiro Martínez D'Elía}
\date{2021}
\maketitle

{\large \textbf{Implemente una solución, paralela, al problema de multiplicación de matrices de $n*n$ con $P<n$}}

Si el número de procesos es menor que \emph{n}, no podremos asignar una fila a cada proceso. En su lugar, debemos hacer que cada proceso trabaje con una porción (\emph{strpe}) del arreglo.

\begin{lstlisting}
Process[w = 1..P]

# El proceso w, procesara las filas first a last
int first = (w-1) * (n/p) + 1;
int last = first + n/p - 1;

for(a_row = first to last)

    for [b_col = 1 to n]
        # Inicializamos la celda acumuladora de C.
        c[a_row, b_col] = 0;

        # Iteramos entre las columnas (A) y filas (B) de interes.
        for [k = 1 to n]
            c[a_row, b_col] = c[a_row, b_col] + (a[a_row, k] * b[k, b_row]);
end;
End;
\end{lstlisting}

\textbf{(a) Suponga n=128 y que cada procesador es capaz de ejecutar un proceso. ¿Cuántas asignaciones, sumas y productos se hacen secuencialmente (caso en el que P=1)?}

\begin{itemize}
    \item En este caso $strip \rightarrow n$
    \item Línea 11: Hace tantas pasadas como columnas tenga B ($n$) por el tamaño del strip $\rightarrow n^2$
    \item Línea 15: Hace $n$ pasadas por tantas pasadas como columnas tenga B por el tamaño del strip $\rightarrow n^3$
\end{itemize}

$Asignaciones = n^3 + n^2 = 128^3 + 128^2 = 2097152 + 16384 = 2113536$

$Sumas = n^3 = 128^3 = 2097152$

$Productos = n^3 = 128^3 = 2097152$


\textbf{(b) Manteniendo n=128. Si los procesadores P1 a P7 son iguales, y sus tiempos de asignación son 1, de suma 2 y de producto 3, y si
P8 es 4 veces más lento, ¿Cuánto tarda el proceso total concurrente? ¿Cuál es el valor del speedup (Tiempo
secuencial/Tiempo paralelo)?. Modifique el código para lograr un mejor speedup.}

\begin{itemize}
    \item En este caso $strip = n/p = 128/8 = 16$
\end{itemize}

$Asignaciones = n^2 \times 16 + n \times 16 = 128^2 \times 16 + 128 \times 16 = 262144 + 2048 = 264192$

$Sumas = n^2 \times 16 = 128^2 \times 16 = 262144$

$Productos = n^2 \times 16 = 128^2 \times 16 = 262144$

Los procesos 1 a 7, tardaran lo mismo:

$264192 \times 1 ut + 262144 \times 2 ut + 262144 \times 3 ut = 1574912 ut$

El proceso 8, es 4 veces más lento que el resto. Por lo cual, tardará 4 veces más:

$1574912 ut \times 4 = 6299648 ut$

Por consiguiente, el proceso concurrente tardará $6299648 ut$ en finalizar. Ya que, el proceso 8 será el último, en terminar su trabajo.

Con las unidades de tiempo, de los procesadores más eficientes, el proceso secuencial tardará:

$2113536 \times 1 ut + 2097152 \times 2 ut + 2097152 \times 3 = 12599296 ut$

Por consiguiente el Speedup obtenido será de 2.

Para mejorar el Speedup podríamos balancear la carga de trabajo, de los procesadores, de manera distinta. Por ejemplo; haciendo que el procesador 8, el más lento, trabaje sobre un strip más pequeño.

\begin{itemize}
    \item Múltiplo de 7 más cercano a 128 $\rightarrow 126$
    \item Tamaño del stripe, para el procesador 8 $\rightarrow 128 - 126 = 2$
    \item Tamaño del stripe, para el resto de los procesadores $\rightarrow 126/7 = 18$
\end{itemize}

Asignaciones P$_{8} = 128^2 \times 2 + 128 \times 2 = 33024$

Sumas P$_{8} = 128^2 \times 2 = 32768$

Productos P$_{8} = 128^2 \times 2 = 32768$

Tiempo P$_8$ $= 33024 \times 1 ut + 32768 \times 2 ut + 32768 \times 3 = 196864 ut$

Asignaciones P$_{resto} = 128^2 \times 18 + 128 \times 18 = 297216$

Sumas P$_{resto} = 128^2 \times 18 = 294912$

Productos P$_{resto} = 128^2 \times 2 = 294912$

Tiempo P$_{resto}= 297216 \times 1 ut + 294912 \times 2 ut + 294912 \times 3 = 1771776 ut$

Con estos nuevos tiempos el speedup será de 7,1.

{\large \textbf{Dado el siguiente programa, indice si es posible que finalice.}}

\begin{lstlisting}
bool continue = true;
bool try = false;
co  while (continue) { try = true; try = false; } #(P)
/   <await(try) continue = false> #(Q)
oc
\end{lstlisting}

Con una política débilmente fair, el programa podría no terminar. Ya que, \emph{try} no se mantiene verdadera hasta ser vista por (Q).

Con una política fuertemente fair, el programa podría terminar. Ya que, \emph{try} se convierte en verdadera con infinita frecuencia.


\end{document}